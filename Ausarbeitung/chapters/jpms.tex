% !TeX spellcheck = de_DE

\section{JPMS}
\label{ch-jpms}

Das folgende Kapitel thematisiert die groben Eigenschaften und Nutzen des \textit{Java Module Platform Systems} (kurz \textit{JPMS}), eingeführt in Java Version 9. 
Das JPMS gilt als eine bzw. \underline{die} zentrale Komponente des \textit{Project Jigsaw} - einen Ansatz die JDK und andere große Code Quellen zu modularisieren. 
Folgende Ziele sollte durch das Project Jigsaw erreicht werden \cite{project-jigsaw}:

\begin{itemize} [noitemsep]
	\item Effektive Zeiteinsparung des Entwicklers bei der Erstellung von größeren Anwendungen und Bibliotheken.
	\item Steigerung von Sicherheit und Wartbarkeit der Java Plattform.
	\item Verbesserung von Anwendungsperformance
	\item Skalierbarkeit von JDK und SE Plattformen, z.B. für den Einsatz in kleineren/embedded Geräten.
\end{itemize}

Diese Ziele wurden größtenteils mit der Einführung des neuen Modulsystems und zwei daraus resultierenden Fähigkeiten in die Java-Welt erfolgreich umgesetzt.
Die neuen Fähigkeiten sind \cite{jpms-spec}:

\begin{itemize}
	\item \textit{Reliable Configuration} (zuverlässige Konfiguration), welches den Standard Classpath Mechanismus ersetzen soll, der unter anderem für die sogenannte \textit{Jar Hell} verantwortlich ist (als Jar Hell werden Probleme, die durch den Classpath Lademechanismus von Java entstehen, bezeichnet). 
	Bei der zuverlässigen Konfiguration müssen sämtliche Abhängigkeiten explizit angegeben werden.
	\item \textit{Strong Encapsulation} (starke Verkapselung) ist die Fähigkeit detailliert zu definieren, auf welche öffentlichen Ressourcen einer Anwendung, externe Komponenten Zugriff erhalten und auf welche nicht.
\end{itemize}

\subsection{Warum Module?}
Welchen Zweck haben diese JPMS Module? Module sind optional. 
Software kann weiterhin im klassischen Stil entwickelt und gewartet werden. 
Java bleibt Java, jedoch bringt die Entwicklung modularer Anwendungen die oben genannten Vorteile mit sich, welche eine Umstellung attraktiv machen.
Im Folgenden werden diese Vorteile genauer beschrieben.

Der offensichtlich wichtigste Vorteil durch die Verwendung des Modulsystems ist die Realisierung einer echte Verkapselung zwischen den einzelnen Modulen.
In konventionellen Java-Anwendungen gibt es keine Möglichkeit, den Zugriff einer Komponente auf Ressourcen einer anderen Kompetente vollständig zu verhindern oder einzuschränken. 
Im Modulsystem hingegen kann für jedes Modul genau festgelegt werden, welche externe Instanzen auf interne Ressourcen eines Moduls zugreifen dürfen.
Diese Zugriffskontrolle ermöglicht eine erhöhte Sicherheit und dürfte somit einige potenzielle Sicherheitslücken schließen. 
Besonders betroffen sind dabei solche, die durch unerlaubten Zugriff auf Ressourcen, die eigentlich geschützt sein sollten, Zustande kommen.

Ein weiterer Vorteil ist der potenzielle Performance-Gewinn.
Da im Voraus eine Übersicht über die für eine Anwendung benötigten Abhängigkeiten definiert ist, kann die JVM zusätzliche Optimierung durchführen, welche ohne eine solche Übersicht nicht realisierbar wären. 
Solche Optimierungen können einen positiven Effekt auf die Beschleunigung von Anwendungen haben.

Ähnlich zum oben genannten Punkt können vorhandene Anwendungen komprimiert werden, da die API ebenfalls als Teil von Project Jigsaw komplett modularisiert wurde. 
Vor Java 9 war die API lediglich in einzelne Pakete aufgeteilt. 
Durch die Modularisierung der API und der Übersicht über die Modulabhängigkeiten werden nur API Module, welche letztendlich für die Anwendung benötigt werden, in die Endsoftware eingebunden. 
Sonstige Module entfallen, wodurch die Anwendungen kleiner werden.

Zusätzlich werden die Anwendungsstrukturen übersichtlicher und die Anwendungen an sich leichter skalierbar als in früheren Java Versionen. 
Ein gutes Beispiel hierfür ist die Java API selbst. 
Vor Java 9 war die API eine Monolith-ähnliche Struktur.
Beim Hinzufügen von zusätzlicher Funktionalität galt es, die strengen und teilweise verschachtelten Abhängigkeiten zu beachten, um eine reibungslose Funktionalität zu gewährleisten. 
Durch die Modalisierung wird die Erweiterung der API stark vereinfacht, da im Zuge dessen lediglich ein neues Modul erstellt wird, das im optimalen Fall von anderen Modulen unabhängig bleibt.
Ebenfalls kann die API z.B. im Bereich von Embedded Systems verkleinert werden, indem Module, die nicht benötigt werden entfernt werden, um wertvolle Ressourcen einzusparen.


\subsection{Module: Was ist ein Java Modul?}
Zunächst wurden die Gründe und Vorteile der Modularisierung betrachtet. 
Im Weiteren wird veranschaulicht, was unter einem JPMS-Modul zu verstehen ist und wie sich diese zu klassischen Java-Anwendungen unterscheiden. 
Folgender Abschnitt geht näher auf die Einzelheiten eines Moduls ein. 
Zusätzlich werden die verschiedenen Modultypen aufgelistet.

\begin{mdframed}[style=mystyle, frametitle=Modul-Definition]
	Ein Modul ist eine Zusammenfassung von eng verknüpften oder stark zusammenhängenden Java-Klassen, Paketen und Ressourcen, zusammen mit einer Deskriptor-Datei, die das Modul definiert bzw. beschreibt.
	Innerhalb eines Moduls können Pakete, wie bisher verwendet werden, um Code zu organisieren.
	Dies ist besonders wichtig, da Pakete das kleinste Granulat für die Zugriffsverwaltung auf die Modul-Ressourcen darstellt.
\end{mdframed}

\subsubsection{Modul-Typen}
Im JPMS gibt es unterschiedlichen Modul-Typen, die hier kurz angerissen werden:

\begin{itemize}
	\item \textit{Plattform Module} (platform modules): Sämtliche Module, die Java SE Ressourcen/Pakete exportieren, werden als Plattform Module (Online auch oft als System Module) bezeichnet.
	
	\item \textit{Unnamed Modul} (unnamed module): Das Unnamed Modul ist ein spezielles Modul, das Rückwärtskompatibilität mit nicht modularisierter Software ermöglichen soll. 
	Beim Versuch, Klassen oder JAR-Dateien zu laden, die sich nicht im Modulpfad befinden, wird zunächst im konventionellen Classpath gesucht. 
	Falls die JAR-Dateien oder Klassen im Classpath geladen wurden, werden sie als Teil des Unnamed Moduls angesehen. 
	Diese werden über das Unnamed Modul zugreifbar.
	
	\item \textit{Automatische Module} (automatic modules): Nicht modularisierte JAR-Dateien, welche im Modulpfad  hinterlegt sind, werden anhand der Namen automatisch Module für diese definiert. 
	Dies ermöglicht es, nicht modularisierte JAR-Dateien wie Module zu verwenden. 
	Sämtliche Ressourcen, von automatischen Modulen sind immer öffentlich zugreifbar.
	
	\item \textit{Anwendungsmodule} (application modules): Von Entwicklern definierte Standard-Module mit Deskriptor-Datei werden auch oft Anwendungsmodule genannt.
\end{itemize}


\subsection{Module: Syntax und Semantik}
Im Anschluss zur Definition folgen Einzelheiten zur Syntax und Semantik von Modulen. 
Zunächst wird die Modul-Deskriptor Datei genauer betrachtet. 

\subsubsection{Modul-Desktiptor: \texttt{module-info.java}}
Die Deskriptor-Datei \texttt{module-info.java} ist ein zentraler Bestandteil für die Erstellung von Modulen.
In dieser Datei werden folgende Eigenschaften festgelegt:

\begin{itemize}[noitemsep]
	\item Name des Moduls.
	\item Abhängigkeiten, welche das Modul besitzt.
	\item Ressourcen-/Paketfreigaben.
	\item Reflexionszugriffsrechte.
	\item Dienste, die das Modul bereitstellt
	\item Dienste, die das Modul verwendet.
\end{itemize}

Als Beispiel werden in Abbildung \ref{fig:mod1-info} und \ref{fig:mod2-info} zwei Modul-Deskriptor-Dateien angegeben, welche in den nachfolgenden Abschnitten genauer erläutert werden.


\begin{figure}
\begin{lstlisting}[language=java]
module com.module1 {
exports com.module1 to com.module0;
exports com.module1.package1 to com.module2;

requires com.module2;
requires com.module3;

opens com.module1 to com.module0;
}
\end{lstlisting}
\caption{Modul1 module-info.java}
\label{fig:mod1-info}
\end{figure}


\begin{figure}
\begin{lstlisting}[language=java]
module com.module1 {
	exports com.module1 to com.module0;
	exports com.module1.package1 to com.module2;
	
	requires com.module2;
	requires com.module3;
	
	opens com.module1 to com.module0;
}
\end{lstlisting}
\caption{Modul2 module-info.java}
\label{fig:mod2-info}
\end{figure}


\subsubsection{Abhängigkeiten}

Im JPMS werden Abhängigkeiten, die einem Modul zugehören mit dem Schlüsselwort \texttt{requires} angegeben. 
Neben \texttt{requires} können zusätzliche Schlüsselwörter, \texttt{static} und \texttt{transitiv}, angegeben werden, die im passenden Kontext verwendet werden können bzw. sollten. 
Es gibt drei Arten mit denen Abhängigkeiten angegeben werden:

\begin{itemize}
	\item \textbf{\texttt{requires}}: 
	Die Standardmethode Abhängigkeiten zu deklarieren.
	Mit \texttt{requires} festgelegte Abhängigkeiten, müssen sowohl zur Kompilierzeit als auch zur Laufzeit vorhanden sein.
	
	\item \textbf{\texttt{requires static}}: 
	Bei statischen Abhängigkeiten handelt es sich um solche, die für Drittanwender eines Moduls optional sind.
	Diese müssen zur Kompilierzeit des Moduls vorhanden sein. 
	
	Diese Art von Abhängigkeiten wird in Fällen verwendet, in denen ein Modul zusätzliche Funktionalitäten bereitstellt, falls eine bestimmte statische Bibliothek vorhanden ist. 
	Es kann vorkommen, dass solche zusätzliche Funktionalitäten für manche Nutzer nicht relevant sind.
	Diese interessieren sich nur für die Kernfunktionalitäten des Moduls. 
	Wären hier sämtliche Abhängigkeiten mit einem normalen \texttt{requires} deklariert, müsste diese vollständig im Modulpfad liegen, was in einem größeren Umfang der Anwendungen resultiert.
	
	Mit \texttt{requires static} gekennzeichnete Abhängigkeiten müssen nicht im Modulpfad liegen, damit ein Nutzer das Modul verwenden kann.
	Entscheidet sich der Nutzer, die zusätzlichen Fähigkeiten des Moduls später zu verwenden, so kann er jederzeit die entsprechende Bibliothek dem Modulpfad hinzufügen und die zusätzlichen Funktionen des Moduls nutzen.
	
	\item \textbf{\texttt{requires transitiv}}: 
	Transitive Abhängigkeiten sind etwas spezieller.
	Es handelt sich um Abhängigkeiten, welche zunächst in einem Modul als transitiv gekennzeichnet werden. 
	Ein Drittmodul, welches das erste Modul einbindet, kann direkt auf die transitive Abhängigkeit zugreifen, ohne diese erneut initialisieren zu müssen.
	
%TODO: Referenz auf Abbildungen evtl.
	
\end{itemize}

\subsubsection{Ressourcenfreigabe}
Gemäß dem Prinzip der starken Verkapselung, werden standardisiert keine Modulressourcen freigegeben. 
Die Freilegung der Modul-API muss explizit in der Desktiptor-Datei angegeben werden.

Mit dem \texttt{exports} Schlüsselwort können alle öffentlichen Klassen eines Pakets zugänglich gemacht werden.
Alles, was sich in anderen Paketen des Moduls befinden, bleibt weiterhin unzugänglich.
Auf Pakete, die mit dem einfachen \texttt{exports} markiert wurden, kann jederzeit zugegriffen werden.
Eventuell soll jedoch verhindert werden, dass der Zugriff auf ein bestimmtes Paket öffentlich ist. 
Eine Kombination von \texttt{exports} mit \texttt{to} kann einer beschränkten Liste von Modulen Zugriff auf das angegebene Paket gewähren.
Der Zugriff von nicht angegebenen Modulen wird verweigert, es sei denn, das gleiche Paket wird zusätzlich mit einem einfachen \texttt{exports} ohne \texttt{to} freigegeben.

\subsubsection{Reflexion}
Reflexion in Java ist die Fähigkeit von Programmen ihre eigene Struktur zur Laufzeit analysieren und verändern zu können.
In Vorgängerversionen vor Java9 konnte man über Reflexion relativ einfach auf sämtliche Ressourcen zugreifen, was unter anderem auch für viele bekannte Java-Sicherheitslücken verantwortlich war.

Einer der Hauptziele vom Modulsystem ist jedoch eine höhere Sicherheit mithilfe von strenger Verkapselung zu erreichen. 
Dies bedeutet, dass es nicht möglich sein sollte, unerwünscht Zugriff auf geschützte Ressourcen zu erhalten.

Somit stehen beide Konzepte, Reflexion und strenge Verkapselung, wie bis jetzt beschrieben, im Widerspruch zueinander. 
Eine Möglichkeit wäre es, Reflexion schlicht zu entfernen (bzw. als veraltet zu markieren), um eine höhere Sicherheit zu garantieren. 
Dies zu tun wäre jedoch eine problematische Entscheidung, da Reflexion eine sehr nützliche und oft verwendete Java-Fähigkeit ist. 

Stattdessen wird man ab Java9 bei Verwendung des Modulsystems - ähnlich zur Ressourcenfreigabe - aufgefordert, für jedes erstellte Modul explizit anzugeben, wer auf welche Ressourcen über Reflexion zugreifen kann. 
Dadurch kann Reflexion weiterhin wie bisher verwendet werden.
Zudem besteht nun die Möglichkeit festzulegen,  wer auf welche Ressourcen über Reflexion zugreifen kann. 

Um ein Modul für reflexiven Zugriff zu öffnen, wird das Schlüsselwort\texttt{open} verwendet.
Dieses gibt das gesamte Modul für Reflexion frei.
Soll jedoch nur beschränkter Zugriff auf bestimmte Ressourcen erlaubt werden, muss stattdessen \texttt{opens} genutzt werden.
Durch die Kombination von \texttt{opens} und \texttt{to} können sowohl Ressourcen als auch externe Module, die Zugriff erhalten sollen, beschränkt werden.


\subsubsection{Dienste}
Dienste (Services) sind ein neues Konzept, welche mit dem Modulsystem einführt wurden.
Grundsätzlich werden zwei Bestandteile unterschieden:

\begin{enumerate}[noitemsep]
	\item Ein Interface oder eine abstrakte Klasse.
	\item Eine oder mehrere Implementierungen dieses Interfaces / dieser abstrakten Klasse. 
\end{enumerate}

Interface und Implementierung befinden sich im Normalfall in separaten Modulen, können jedoch auch im gleichen Modul vorliegen.


\begin{figure}[hbt!]
\begin{lstlisting}[language = java, firstnumber = 1 ,]
module service.implementation {
	requires module.service;
	provides module.service.Service with service.implmentation.MyImplementation;
}
\end{lstlisting}
\caption{Modul2 module-info.java}
\label{fig:module2-info}
\end{figure}


Um eine Implementierung eines Dienstes bereitzustellen, werden \texttt{provides} und \texttt{with} verwendet.
\texttt{prodives} gibt das bzw. den zu implementierende Interface bzw. Dienst an. 
\texttt{with} wird verwendet, um eine Implementierung dieses Dienstes anzugeben und für Konsumenten-Module bereitzustellen. 
Im obigen Beispiel wird eine Implementierung für das \texttt{module.service.Service} Interface angegeben und bereitgestellt.


\begin{figure}[hbt!]
\begin{lstlisting}[language = java, firstnumber = 1 , escapeinside={(*@}{@*)}]
module service.user {
	requires module.service;
	uses module.service.Service;
}
\end{lstlisting}
\caption{Service Implementierung Nutzen}
\label{fig:service-use}
\end{figure}


Um einen Dienst einzubinden, muss  dieser zunächst als Abhängigkeit angeben werden, wie in Abbildung \ref{fig:service-use} zu sehen ist.
Hier ist zu beachten, dass das Implementierungsmodul als Abhängigkeit deklarieren werden muss, es sei denn, das zu implementierende Interface befindet sich im gleichen Modul wie die Implementierung.
%TODO überprüfen
Schlussendlich wird der Dienst, der eingebunden wird mit \texttt{uses} deklariert.

Ein Vorteil, den man hierdurch hat, ist der, dass die Implementierung des Interfaces auswechselt werden kann, ohne dass der Client-Code geändert werden muss.

% TODO Add Link https://www.baeldung.com/java-9-modularity